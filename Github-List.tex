%-------------------------
% Resume in Latex
% Author : Sourabh Bajaj
% License : MIT
%------------------------

\documentclass[letterpaper,11pt]{article}

\usepackage{latexsym}
\usepackage[empty]{fullpage}
\usepackage{titlesec}
\usepackage{marvosym}
\usepackage[usenames,dvipsnames]{color}
\usepackage{verbatim}
\usepackage{enumitem}
\usepackage[hidelinks]{hyperref}
\usepackage{fancyhdr}
\usepackage[english]{babel}

\usepackage{ragged2e}

\pagestyle{fancy}
\fancyhf{} % clear all header and footer fields
\fancyfoot{}
\renewcommand{\headrulewidth}{0pt}
\renewcommand{\footrulewidth}{0pt}

% Adjust margins
\addtolength{\oddsidemargin}{-0.5in}
\addtolength{\evensidemargin}{-0.5in}
\addtolength{\textwidth}{1in}
\addtolength{\topmargin}{-.2in}
% \addtolength{\topmargin}{-.5in}
\addtolength{\textheight}{1.0in}

\urlstyle{same}

\raggedbottom
\raggedright
\setlength{\tabcolsep}{0in}

% Sections formatting
\titleformat{\section}{
  \vspace{0pt}\scshape\raggedright\large
}{}{0em}{}[\color{black}\titlerule \vspace{-5pt}]

%-------------------------
% Custom commands
\newcommand{\resumeItem}[2]{
  \item\small{
    \textbf{#1}{#2 \vspace{-2pt}}
    % \textbf{#1}{: #2 \vspace{-2pt}}
  }
}


\newcommand{\resumeSubheading}[4]{
  \vspace{-1pt}\item
    \begin{tabular*}{0.97\textwidth}[t]{l@{\extracolsep{\fill}}r}
      \textbf{#1} & #2 \\
      \textit{\small#3} & \textit{\small #4} \\
    \end{tabular*}\vspace{-5pt}
}

\newcommand{\resumeSubheadingOneLine}[2]{
  \vspace{-1pt}\item
    \begin{tabular*}{0.97\textwidth}[t]{l@{\extracolsep{\fill}}r}
      \textbf{#1} & \textit{\small #2} \\
    \end{tabular*}\vspace{-5pt}
}

\newcommand{\resumeSubheadingOneLineNoB}[2]{
  \vspace{-1pt}\item
    \begin{tabular*}{0.97\textwidth}[t]{l@{\extracolsep{\fill}}r}
      #1 & \textit{\small #2} \\
    \end{tabular*}\vspace{-20pt}
}

\newcommand{\resumeSubItem}[2]{\resumeItem{#1}{#2}\vspace{-4pt}}

\renewcommand{\labelitemii}{$\circ$}

\newcommand{\resumeSubHeadingListStart}{\begin{itemize}[leftmargin=*]}
\newcommand{\resumeSubHeadingListEnd}{\end{itemize}}
\newcommand{\resumeItemListStart}{\begin{itemize}}
\newcommand{\resumeItemListEnd}{\end{itemize}\vspace{-5pt}}

%-------------------------------------------
%%%%%%  CV STARTS HERE  %%%%%%%%%%%%%%%%%%%%%%%%%%%%


\begin{document}

%----------HEADING-----------------
\begin{tabular*}{\textwidth}{l@{\extracolsep{\fill}}r}
  \textbf{\Large Lanxin Lei} & {}\\
  % \textbf{\href{http://sourabhbajaj.com/}{\Large Lanxin Lei}} & Email : \href{mailto:sourabh@sourabhbajaj.com}{sourabh@sourabhbajaj.com}\\
  % \href{http://sourabhbajaj.com/}{http://www.sourabhbajaj.com} & Mobile : (+86) 15332370878  \\
  % {University of Electronic Science and Technology of China} & Mobile : (+86) 15332370878  \\
  % {Embedded specialty. Graduation date: \small{30/06/2019}} & {\small GPA} : 3.49/4  \\
\end{tabular*}


% %-----------EDUCATION-----------------
% \section{Education}
%   \resumeSubHeadingListStart
%     \resumeSubheading
%       {Georgia Institute of Technology}{Atlanta, GA}
%       {Master of Science in Computer Science;  GPA: 4.00}{Aug. 2012 -- Dec. 2013}
%     \resumeSubheading
%       {Birla Institute of Technology and Science}{Pilani, India}
%       {Bachelor of Engineering in Electrical and Electronics;  GPA: 3.66 (9.15/10.0)}{Aug. 2008 -- July. 2012}
%   \resumeSubHeadingListEnd

% -----------PROFILE SUMMARY -----------------
\section{List of projects made in mmlab}

      \resumeItemListStart

        \resumeItem{Exploration more for potentially good actions} \\
        \begin{itemize}
          \item \justifying{Implemented the method that made use of shorter estimations to  encourage actions with ”potential” of being good.}
          \begin{itemize} \item \href{https://github.com/LanxinL/minMaxExp}{https://github.com/LanxinL/minMaxExp} \end{itemize}
          \begin{itemize} \item \href{https://github.com/LanxinL/minMaxExp/invitations}{Invitation link: https://github.com/LanxinL/minMaxExp/invitations} \end{itemize}
          \item \justifying{Preparing for submitting the project to NeurIPS 2019.}     
        \end{itemize}

        \resumeItem{\href{}Exploit in exploration} \\
        \begin{itemize}
          \item\justifying{Implemented the method that used the signs of advantage value to train a discriminator for improving sampling efficiency of policy, trained in the Mujoco environments.}
          \begin{itemize} \item \href{https://github.com/LanxinL/GanRl/tree/repo/gan_fixStd_extra_item}{https://github.com/LanxinL/GanRl/tree/repo/gan\_fixStd\_extra\_item} \end{itemize}
          \begin{itemize} \item \href{https://github.com/LanxinL/GanRl/invitations}{Invitation link: https://github.com/LanxinL/GanRl/invitations} \end{itemize}
            \item \justifying{Submission of ICML 2019.} 
        \end{itemize}

        \resumeItem{\href{}Sub-Environments} \\
        \begin{itemize}
          \item\justifying{Created two sub-environments for Arm3D environment to make reward designing easier.}
          \begin{itemize} \item \href{https://github.com/LanxinL/arm3dENV}{https://github.com/LanxinL/arm3dENV} \end{itemize}          
          \begin{itemize} \item \href{https://github.com/LanxinL/arm3dENV/invitations}{Invitation link: https://github.com/LanxinL/arm3dENV/invitations} \end{itemize}
            \item\justifying{Implemented the method that parallelly trained both the sub-environments and the initial environment.}
          \begin{itemize} \item \href{https://github.com/LanxinL/arm3d}{https://github.com/LanxinL/arm3d} \end{itemize}
          \begin{itemize} \item \href{https://github.com/LanxinL/arm3d/invitations}{Invitation link: https://github.com/LanxinL/arm3d/invitations} \end{itemize}
            % \item\justifying{Oue method divides the initial Arm3D environment into two sub-environments, and parallelly trains both the sub-environments and the initial environment, which makes the reward designing easier. }
          \item {Validated the effectiveness of the method in a toy simulator, a grid world with one single line.}
          \begin{itemize} \item \href{https://github.com/LanxinL/Q-learning-Gridworld}{https://github.com/LanxinL/Q-learning-Gridworld} \end{itemize}
          \begin{itemize} \item \href{https://github.com/LanxinL/Q-learning-Gridworld/invitations}{Invitation link: https://github.com/LanxinL/Q-learning-Gridworld/invitations} \end{itemize}
          \end{itemize}

        \resumeItem{\href{}Visualization tool} \\
        \begin{itemize}
          % \item\justifying{For each parameters' setting, the tool can draw its outputs under the different seeds into a single curve, then makes the comparison of performance easier.}
          \item\justifying{For the outputs of all the seeds under the same parameters' setting, the tool can draw them into a single curve, which makes the comparison of performance easier.}
          \begin{itemize} \item \href{https://github.com/LanxinL/plotViaSeaborn}{https://github.com/LanxinL/plotViaSeaborn} \end{itemize}
          \begin{itemize} \item \href{https://github.com/LanxinL/plotViaSeaborn/invitations}{Invitation link: https://github.com/LanxinL/plotViaSeaborn/invitations} \end{itemize}
          \end{itemize}
        
      \resumeItemListEnd

  

% %-----------PROJECTS-----------------
% \section{Projects}
%   \resumeSubHeadingListStart
%     \resumeSubItem{QuantSoftware Toolkit}
%       {Open source python library for financial data analysis and machine learning for finance.}
%     \resumeSubItem{Github Visualization}
%       {Data Visualization of Git Log data using D3 to analyze project trends over time.}
%     \resumeSubItem{Recommendation System}
%       {Music and Movie recommender systems using collaborative filtering on public datasets.}
%     \resumeSubItem{Mac Setup}
%       {Book that gives step by step instructions on setting up developer environment on Mac OS.}
%   \resumeSubHeadingListEnd

%
%--------PROGRAMMING SKILLS------------
%\section{Programming Skills}
%  \resumeSubHeadingListStart
%    \item{
%      \textbf{Languages}{: Scala, Python, Javascript, C++, SQL, Java}
%      \hfill
%      \textbf{Technologies}{: AWS, Play, React, Kafka, GCE}
%    }
%  \resumeSubHeadingListEnd


%-------------------------------------------
\end{document}
